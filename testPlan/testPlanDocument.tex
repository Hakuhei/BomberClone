\documentclass[12pt, letterpaper]{article}
\usepackage[utf8]{inputenc}

\usepackage{graphicx}
\graphicspath{{images/}}

\title{\textbf{Test Plan for Bomber Clone Application \\ \Large Software Engineering 3XA3: Project}}
\author{Group 12 - The A Team \\Gabriel Lopez De Leon, 1310514\\Ren-David Dimen, 1222679\\Jay Nguyen, 1327828}
\date{23 October 2015}

\renewcommand*\contentsname{Table of Contents}
\setcounter{tocdepth}{4}
\setcounter{secnumdepth}{4}

\begin{document}
	
	\begin{titlepage}
		\clearpage\maketitle
		\thispagestyle{empty}
	\end{titlepage}
	
	\newpage
	\tableofcontents
	\newpage
	
	\section{Revision History}
	
		\begin{tabular}{ |c|c|c|c| } 
			\hline
			\textbf{Revision \#} & \textbf{Revision Date} & \textbf{Description of Change} & \textbf{Author}\\
			\hline
			8 & Oct 23 2015 & Added Section 17 & Ren-David Dimen\\
			\hline
			7 & Oct 23 2015 & Added Section 16 and 18 & Gabriel Lopez de Leon\\
			\hline
			6 & Oct 23 2015 & Added Section 5 & Ren-David Dimen\\
			\hline
			5 & Oct 23 2015 & Added Section 7-15 & Gabriel Lopez de Leon\\
			\hline
			4 & Oct 22 2015 & Update Section 3 & Gabriel Lopez de Leon\\
			\hline
			3 & Oct 22 2015 & Added Sections 4 and 6 & Gabriel Lopez de Leon\\
			\hline
			2 & Oct 21 2015 & Added Sections 1-3 & Gabriel Lopez de Leon\\
			\hline
			1 & Oct 21 2015 & Created Test Plan Document & Gabriel Lopez de Leon\\
			\hline
		\end{tabular}
	
	\newpage
	
	\section{General Information}
	\indent \indent The following section will provide an overview of the Bomber Clone test plan. This section explains the main purpose of the document, the test plan identifier, any key terms and acronyms/abbreviations, and any references.
	
	\subsection{Introduction}
	\indent \indent This document is the main test plan for the Bomber Clone Application which is a recreation of legacy code for a Bomberman game. This plan will address possible software risks and the features of the game that are to be tested. The main targets for testing include the game states and player states while ensuring there are minimal to no bugs when running the program.\\
	
	This project will be tested through both unit testing and manual tests of how well the game interacts with user inputs. The details on the two levels of testing will be adressed in further depth in the later sections of this document.
	
	\subsection{Test Plan Identifier}
	Bomber Clone Master Test Plan: \textbf{Version 0}
	
	\subsection{Key Terms, Acronyms, and Abbreviations}

		\begin{tabular}{ |c|c|c|c| } 
			\hline
			\textbf{Symbol} & \textbf{Description} \\
			\hline
			QA & Quality Assurance\\ 
			SRS & Software Requirements Specification\\ 
			PoC & Proof of Concept Plan\\ 
			GUI & Graphic User Interface\\
			UI & User Interface\\
			OS & Operating System\\
			IDE & Integrated Development Environment\\
			TA & Teaching Assistant\\
			\hline
		\end{tabular}
	
	\subsection{References}
	\indent \indent This document references the IEEE Test Plan Outline, our team's SRS, and PoC which can all be found online or on gitlab.
	
	\section{Test Items}
	The following is a list of key aspects of the program to be tested:\\
	
	\noindent A. \indent Game State: which checks if the game has started, if a player has won, and if the board is setup correctly when the game begins.\\
	
	\noindent B. \indent Player State: which checks if the player has spawned at the start of the game, if the player is still alive or dead, and if the player gains a power-up.\\
	
	\noindent Note: the above test items are meant to be tested using JUnit automated testing. The following are manual tests to test aspects of the game which cannot be checked by automated testing.\\ 
	
	\noindent C. \indent Game Functionality: testing if the user interfce works properly and how well the game responds to user inputs.
	
	\subsection{Functional Requirements}

	\noindent \textbf{Test Type}: Unit Testing (Automated)\\
	\noindent \textbf{Summary}: The system shall keep track of the state of the arena.\\
	\noindent \textbf{Test Factors}: Correctness, reliability and continuity of process.\\
	\noindent \textbf{Rationale}: To update the game as it progresses.\\

	\noindent \textbf{Test Type}: Unit Testing (Automated)\\
	\noindent \textbf{Summary}: The game shall have a timer which tracks how long before the game ends.\\
	\noindent \textbf{Test Factors}: Reliability and continuity of process. \\
	\noindent \textbf{Rationale}: To limit the time for each game.\\

	\noindent \textbf{Test Type}: Black Box Test (Manual)\\
	\noindent \textbf{Summary}: The arena blocks must spawn at random.\\
	\noindent \textbf{Test Factors}: Correctness and reliability.\\
	\noindent \textbf{Rationale}: To add diversity between games, as well as change up the destructible/indestructible blocks.\\

	\noindent \textbf{Test Type}: White Box Test (Manual)\\
	\noindent \textbf{Summary}: The game shall be able to take multiple inputs.\\
	\noindent \textbf{Test Factors}: Performance, correctness, useability and reliability.\\
	\noindent \textbf{Rationale}: To allow for multiple players to use one keyboard.\\
	
	\subsection{Non-Functional Requirements}
	\indent \indent The main non-functional requirements that are to be tested include correctness, reliability, and performance. The game should be able to run properly and execute its functionality correctly and consistently. The performance of the program overall is also very important and needs to be taken into account, especially with regards to how well the game accepts user inputs.
	
	\section{Plan}
	\indent \indent The following section provides a description of the software being tested, the team that will perform the testing, the milestones for the testing phase, and the budget allocated to the testing.
	
	\subsection{Software Description}
	\indent \indent The software being tested is a re-implementation of legacy java code which simulates bomberman the game. The game is played with one or two players with the possible addition of computer players which will be implemented at a later time. A time limit will be added to the game to limit the length of each match. Power-ups will be an additional feature to be added later on, outside the project timeline for this course.
	
	\subsection{Test Team}
	\indent \indent The team that will execute the test cases, write and review the test plan/report consists of : Gabriel Lopez de Leon, Ren-David Dimen, Jay Nguyen.
	
	\subsection{Milestones}
	
	\subsubsection{Location}
	The location where the testing will be performed is Hamilton, Ontario. The institution that will be performing the testing is The A Team, group 12 in McMaster University's Software Engineering 3XA3 Lab Section L01.
	
	\subsubsection{Dates and Deadlines}
	Test Case:\\
	\noindent The creation of the test cases for both system testing and unit testing have begun already as of October 21, 2015.\\
	
	\noindent Test Case Completion:\\
	\noindent Implementing code for automated testing such as JUnit is scheduled to begin immediately and to continue throughout the software development process. The test cases should be completed by the Test Report due on November 27, 2015.\\
	
	\noindent Test Report Revision 0:\\
	\noindent The Test Report Revision 0 is to be written by November 27, 2015.
	
	\subsection{Budget}
	\indent \indent There is no budget for the testing of the system, the only constraint for testing would be in regards to the time it would take to implement the various tests.
	
	\section{Software Risk Issues}
	\indent \indent The game is designed to run on the user's personal computers and as such may present some risks when running on the device or within the code itself. In some instances the following software issues may arise and should be tested for.\\
	
	\noindent 1. \indent The program may contain an infinite loop that may deplete the\\
	\indent \indent capabilities of the device it is running on. This may cause the game\\
	\indent \indent to crash or in a more serious situation, cause the computer to freeze.\\
	
	\noindent 2. \indent The game is built to have many sections of the program to be updating\\
	\indent \indent and running at once. This may create a case of deadlock. If this were\\
	\indent \indent to happen, the game may stop running.\\
	
	\noindent 3. \indent Both the user and the game's memory will be updating the board\\
	\indent \indent displayed. This presents a possibility of interference between what\\
	\indent \indent the user and what the memory is sending to the board resuting in an\\ 
	\indent \indent inaccurate output.\\
	
	\noindent 4. \indent Some components in the game's programming will be asynchronous.\\
	\indent \indent As such, the game may take some time to update sections of the code\\
	\indent \indent causing a delay in the updating of the board state or it may interfere\\
	\indent \indent with other data stored within the game's memory.\\
	
	\noindent 5. \indent The game is written so that most modern computers may run it\\
	\indent \indent however, there may exist some devices that may not meet the\\
	\indent \indent software requirments to run the sysyem.\\
	
	\noindent 6. \indent Several packages will be used in the implementation of the program.\\
	\indent \indent All packages should be executable when running the program.
	
	\section{Features To Be Tested}
	The following is a list of the areas to be focused on during testing of the applicaiton:\\
	\noindent A. \indent Addition of a game timer.\\
	\noindent B. \indent Redesigned/improved implementation of the board and possible game \indent \indent states.\\
	\noindent C. \indent Redesigned/improved inputs, and controls in game. (compared to legacy code)\\
	
	\section{Features Not To Be Tested}
	The following is a list of areas that will not be specifically addressed. All testing in these areas will be indirect as a result of other testing efforts:\\
	
	\noindent A. \indent Performance on Various OS\\
	
	This non-functional requirement is not one of the main priorities during the testing phase and is not a key aspect that needs to be implemented. Though it is ideal if the program is able to run on various OS, it is not the main focus of the project.\\
	
	\noindent B. \indent GUI and their Functionality\\
	
	The GUI such as menus, windows, and buttons are assumed to be implemented properly and will be tested indirectly while testing the overall program.
	
	\section{Approach}
	\subsection{Test Tools}
	\indent \indent The main test tool used during the testing period will be JUnit which is built into the Eclipse IDE for Java programming.
	
	\subsection{Meetings}
	\indent \indent Meetings to work on the assignment will be mainly during lab periods each week, any further meetings to work collectively as a group will take place when members see it fit. However, most of the files needed for the project are being shared between group members through GitLab online, so working together on the project does not always require members to be with each other in person.
	
	\subsection{Testing Levels}
	
	\subsubsection{Types of Tests}
	\indent \indent There are various types of testing which will be used throughout this testing process. The various types of tests include, white box, black box, unit, manual and automated testing. White box testing will be used mainly to test functionality of complex methods and methods which relate to gameplay, ones that cannot be tested using automated testing. Block box testing will be used for simpler functions to see if they work or not, disregarding how it is implemented. Most of the white box and black box testing will be done manually, and JUnit testing will be automated through the Eclipse IDE.
	
	\subsubsection{Game State Testing}
	\indent \indent The testing for our group's game project will consist of the various types of tests above with a main focus on automated testing. The test team, which consists of all the members of group 12, will all test parts of the code throughout its implementation and will be revised/approved of by the other members of the team. Proof of the tests will be seen through documentation and through JUnit files which will also be uploaded into the GitLab repository. Throughtout the testing process, the test team will be relying on JUnit as the main test tool for unit testing.\\
	
	Our group will be using unit testing to test the various possible game states which occur as the program runs. The rationale behind using unit testing is because we feel that using automated testing for a key aspect of the game will be more efficient and allows for consistent checks to see if the reliability and continuity of process requirements are still being met. Furthermore, using unit testing provides instant feedback when the JUnit file is run. The user is immediately notified by a green light if the expected output is still being outputted after any changes while as a red light would mean that the functionality that the specific unit test is checking is no longer producing the correct output. Consistent testing of the state of the game is important as it is the backbone of the application, the game relies on this to check if a match has begun, how the arena changes and if a player has won.\\
	
	During the proof of concept demonstration, the game state will be tested by presenting a working implementation of the game which should show how the requirements are addressed and how the unit tests result in a positive outcome (green light for JUnit tests). The inputs for the unit tests will be random placements of bombs in the arena with an expected output of them affecting the board. Another test will take an input of a bomb placement to eliminate the enemy player where the expected output should be a player win. These tests will be performed by the whole test team and will be demonstrated to the TA.
	
	\subsubsection{Player State Testing}
	\indent \indent Player state will also be tested using unit testing as it is another very important part of the program which needs to be consistently checked to ensure it works correctly and with its intended functionality. The state of the player must be tested with JUnit as immediate feedback is once again necessary. The state of the player can change at anytime during a match, so the system must keep checking for the key states such as if the player has spawned correctly at the start of the game, if the player has collected a power-up, and if the player is alive or dead. This testing can be done by taking in an input of the game starting with expected output as the player appearing on the arena. Another input would be placing a bomb next to a player to see if the player's state changes upon the bomb exploding.
	
	\subsubsection{General Functionality}
	\indent \indent Our group also plans on testing the overall functionality of the game manually by running and playing the game to ensure all the features are present and everything works as intended. A description of how to test the program manually will likely be added later on and updated throughout the software development process. Overall, these tests are simple to execute and only take the user input from keyboard listeners and will require the user to check and see if the game did what it is supposed after a certain key is pressed or after a certain interaction occurs.
	
	\section{Item Pass/Fail Criteria}
	\indent \indent With respect to the unit testing for the Game and Player State, the test will be passed once the Eclipse IDE give the green light signaling that the expected output matches the actual output for the given unit test. JUnit code will be what tests to see if it passes or not as it validates the output. As for other testing, mainly referring to the manual tests, these tests pass if the game responds well to user input and translates it to an action in game which it is supposed to do.
	
	\section{Suspension and Resumption of Testing}
	\indent \indent Testing will be taking place throughout the software development process and no suspension or delay of testing are expected. The only possible criteria for suspension include the delay time when the program is being coded and is unable to be run. The resumption requirement would simply be to complete the code enough to allow for it to run so that various tests either manual or automated can be completed.
	
	\section{Test Deliverables}
	The following is a list of the deliverable related to testing:\\
	\noindent A. \indent Test Plan Revision 0\\
	\noindent B. \indent Test Report Revision 0\\
	\noindent C. \indent Test Plan Revision 1\\
	\noindent D. \indent Test Report Revision 1
	
	\section{Remaining Tasks}

		\begin{tabular}{ |c|c|c|c| } 
			\hline
			\textbf{Task} & \textbf{Assigned To} & \textbf{Status}\\
			\hline
			Complete Test Plan Revision 0 & Test Team & In Progress\\
			\hline
			Create Test Report Revision 0 & Test Team & Not Started\\
			\hline
			Create Test Plan Revision 1 & Test Team & Not Started\\
			\hline
			Create Test Report Revision 1 & Test Team & Not Started\\
			\hline
			Complete Proof of Concept Demo & Group 12 & In Progress\\
			\hline
		\end{tabular}\\\\
		
		\noindent Note: Test Team and Group 12 consist of the same members but  the two names are used to identify the type of the task, whether it relates to testing, programming or the overall project.
	
	\section{Environmental Needs}
	\indent \indent Not much is needed to support testing in our project, the main requirement would be to ensure that the JUnit framework is installed for unit testing in the Eclipse IDE. The rest of the testing mainly requires the test team to just run the program and see the results for black box tests and use manual tests for some of the white box testing.
	
	\section{Staffing and Training Needs}
	\indent \indent Further training for this project is not necessary, it is within the scope of capabilites of the project team. Each member of the test team have experience with Java programming, the Eclipse IDE and unit testing using JUnit. With regards to staffing, each member of the test team will be required to consistently run tests on the code as it is being implemented.
	
	\section{Responsibilities}
	\indent \indent There is no set leader of the project or testing phase, each group member is expected to contribute and work together to complete the tests. All of the work done in the testing period will be split amongst the test team ensuring that all manual and automated tests are completed and properly documented.
	
	\section{Schedule}
	The following is a description of the allocated time for different testing activities:\\
	
	\noindent A. \indent Development of master test plan Revision 0: due on October 23, 2015\\
	\noindent B. \indent Review of test plan before implementing tests: scheduled for October \indent \indent 24/25, 2015\\
	\noindent C. \indent Review of SRS and PoC to ensure the code is implemented correctly: \indent \indent scheduled for October 24/25, 2015\\
	\noindent D. \indent Break in the testing process to complete most of code implementation: \indent \indent scheduled for November 19-23,  2015\\
	\noindent E. \indent Unit testing: due by Test Report, on the November 27, 2015
	
	\section{Planning Risks and Contingencies}
	
	\noindent A. \indent Limited Working Staff\\
	\\
	\indent The staff assigned to complete the project is extremely limited and losing or having a member reassigned would severely hinder the progress of the project.\\
	\indent Should this occur, the Project Supervisor and the Professor will be updated on the situation and updates to the project's deadlines and requirements will be adjusted accordingly.\\
	
	\noindent B. \indent The Scope of the Project is too Big\\
	\\
	\indent The desired results of the program as desired by the client may be larger than feasible given the time constraint. As such, it may be difficult to complete the project as required.\\
	\indent To help minimize this risk, tasks will be alloted a specific amount of time to which it shall be completed. If the tasks can not be completed in time, the working team shall review the tasks and see what is essential to the project and what can be removed. This shall help simplify what is needed to be done.\\
	
	\noindent C. \indent Technical Complexity\\
	\\
	\indent As the programmers are familiar with the project in question, it may seem like it will be simple to implement. Underestimating the complexity of the task at hand may lead to simple errors in the program.\\
	\indent To combat this problem, the program will be periodically compiled and ran to ensure that the project is up to standard.
	
	\section{Approvals}
	
		\begin{tabular}{ |c|c|c|c| } 
			\hline
			Project Supervisor - Hediyeh Motaghian (TA) & \indent \indent \indent \indent \indent \indent \indent \indent \indent\\
			\hline
			Professor - Spencer Smith &\\
			\hline
		\end{tabular}
	
\end{document}